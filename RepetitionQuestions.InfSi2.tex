%Pakete;
%A4, Report, 12pt
\documentclass[ngerman,a4paper,12pt]{scrreprt}
\usepackage[a4paper, right=20mm, left=20mm,top=20mm, bottom=30mm, marginparsep=5mm, marginparwidth=5mm, headheight=7mm, headsep=15mm,footskip=15mm]{geometry}

%Papierausrichtungen
\usepackage{pdflscape}
\usepackage{lscape}

%Deutsche Umlaute, Schriftart, Deutsche Bezeichnungen
\usepackage[utf8]{inputenc}
\usepackage[T1]{fontenc}
\usepackage[ngerman]{babel}

%quellcode
\usepackage{listings}

%tabellen
\usepackage{tabularx}

%listen und aufzählungen
\usepackage{paralist}

%farben
\usepackage[svgnames,table,hyperref]{xcolor}

%symbole
\usepackage{latexsym,textcomp}

%font
\usepackage{helvet}
\renewcommand{\familydefault}{\sfdefault}

%Abkürzungsverzeichnisse
\usepackage[printonlyused]{acronym}

%Bilder
\usepackage{graphicx} %Bilder
\usepackage{float}	  %"Floating" Objects, Bilder, Tabellen...
\usepackage[space]{grffile} %Leerzechen Problem bei includegraphics
\usepackage{wallpaper} %Seitenhintergrund setzen
\usepackage{transparent} %Transparenz

%for
\usepackage{forloop}
\usepackage{ifthen}

%Dokumenteigenschaften
\title{Repetitionsfragen InfSi2}
\author{Tobias Blaser}
\date{\today{}, Rapperswil}


%Kopf- /Fusszeile
\usepackage{fancyhdr}
\usepackage{lastpage}

\pagestyle{fancy}
	\fancyhf{} %alle Kopf- und Fußzeilenfelder bereinigen
	\renewcommand{\headrulewidth}{0pt} %obere Trennlinie
	\fancyfoot[L]{Seite \thepage/\pageref{LastPage}} %Fusszeile mitte
	\fancyfoot[R]{\today{}} %Fusszeile rechts
	\renewcommand{\footrulewidth}{0.4pt} %untere Trennlinie

%Kopf-/ Fusszeile auf chapter page
\fancypagestyle{plain} {
	\fancyhf{} %alle Kopf- und Fußzeilenfelder bereinigen
	\renewcommand{\headrulewidth}{0pt} %obere Trennlinie
	\fancyfoot[L]{Seite \thepage/\pageref{LastPage}} %Fusszeile mitte
	\fancyfoot[R]{\today{}} %Fusszeile rechts
	\renewcommand{\footrulewidth}{0.4pt} %untere Trennlinie
}

\usepackage{changepage}

% Abkürzungen für Kapitel, Titel und Listen
\input{commands/shortcutsListAndChapter}
\input{commands/TextStructuringBoxes}

%links, verlinktes Inhaltsverzeichnis, PDF Inhaltsverzeichnis
\usepackage[bookmarks=true,
bookmarksopen=true,
bookmarksnumbered=true,
breaklinks=true,
colorlinks=true,
linkcolor=black,
anchorcolor=black,
citecolor=black,
filecolor=black,
menucolor=black,
pagecolor=black,
urlcolor=black
]{hyperref} % Paket muss unbedingt als letzes eingebunden werden!

\usepackage{graphicx}
\begin{document}

% Inhaltsverzeichnis
\tableofcontents
\clearpage

\ch{Kryptographische stärke}
\ol
	\li Abb. \ref{csnt}: Füllen Sie die Tabelle aus.
	\img{img/r1.1.jpg}{}{1}{csnt}
	\li Was ist das Ziel der NSA Suite? Wozu dienen Suite A und B?
	\li Erklären Sie wie das Prinzip von Elliptic Curves funktioniert.
	\li Wie wird mit EC ``Secret Key Exchange'' gemacht?
	\li Was sind ECDH, ECIES und ECDSA?
	\li Was ist AEAD, welchen Vorteil bringt es?
\olS


\ch{Physical Layer Security}
\olR
	\li Erklären Sie die beiden veralteten Sicherheitsverfahren: Voice Scrambling, Frequency Hopping
	\li Wie funktioniert Key Distribution mittels Quanten?
	\li Wie kann bei Quanten festgestellt werden, wenn jemand mithört?
	\li Wie funktioniert key Distribution mit Photonen nach BB84? Wie wird hier verhindert, das jemand mithört? Welche praktischen Einschränkungen besitzt die Technologie?
	\li Welche technischen Schwierigkeiten gibt es heute grundsätzlich mit Photonendatenübertragung?
	\li Warum braucht ein Sicherheitssystem mit Photonen/Quanten immer zwei Kommunikationskanäle?
\olS


\ch{Schlüsselmaterial und Zufallszahlen}
\olR
	\li Erklären Sie, wie HMAC RFC 2104 funktioniert.
	\li Erklären Sie, was PRF ist, wie es funktioniert und warum er generierte Key Stream keine grössere Entropie aufweisen kann als das ursprüngliche Secret.
	\li Warum wird die Berechnung des TLS 1.1 Master secrets auf zwei Teile (MD5/SHA-1) aufgeteilt? Wie wird bei TLS 1.1 Schlüsselmaterial generiert'
	\li Welche Möglichkeiten gibt es, um richtige Zufallszahlen zu generieren?
	\li Was sind Hardware basierte Zufallszahlengeneratoren?
	\li Wie funktioniert der Quantum Random Number Generator von IDQ?
	\li Wie können Sie die statistische Verteilung von Zufallsmaterial verändern (von Neumann)?
\olS


\ch{Data Link Layer Security}
\olR
	\li Erklären Sie wie die Authentisierung bei EAP mit Radius Server funktioniert.
	\li Was ist die Secure Device ID (DecID)? Wozu kann sie verwendet werden? Können Sie die DevID verändern? begründen Sie. Welche weiteren Informationen / Elemente sollte das DevID Modul noch beinhalten?
	\li Was ist eine Connectivity Association? Welchen Vorteil bieten Secure Channel und Secure Association?
	\li Wie ist das MACsec Frame Format aufgebaut? Erklären Sie die Felder SecTag, Secure Data, ICV.
	\li Wie funktioniert die Key Distribution mit MKA, SAK und CAK?
	\li Wie können Sie mit EAP dynamische CAK's ermöglichen?
\olS


\ch{Application Security}
\olR
	\li Was ist das OWASP?
	\li Nennen Sie die 5-A Prinzipien, die bei Web Application Security zum Tragen kommen.
	\li Erklären Sie, wie eine 3-Tier Web Application aufgebaut ist und welche Security Massnahmen sie auf jedem Tier vornehmen müssen.
	\li Was ist Clickjacking?
	\li Was ist Cross-Site Scripting? Erklären Sie die drei Arten.
	\li Fassen sie kurz den Whitehat Security Report zusammen.
	\li Notieren Sie typische CIA-Sicherheitsprobleme bei Webanwendungen (Chat Vorlesung).
	\li Was sind die OWASP Top 10. Nach welchen Kriterien werden sie bestimmt? Nenen Sie die ersten fünf.
	\li Erklären Sie das bereits aus InfSi1 bekannte OWASP Thread Model.
	\li Was sind Vulnerability Factors? Nennen Sie einige.
	\li Nennen Sie die fünf Phasen des OWASP Security Testing Guide.
	\li Nennen Sie die wichtigsten Punkte aus dem OWASP Application Security Architecture Cheat Sheet.
	\li Erklären Sie den Thread: Insufficient TLS Protection
\olS






\end{document}
