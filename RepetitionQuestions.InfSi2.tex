%Pakete;
%A4, Report, 12pt
\documentclass[ngerman,a4paper,12pt]{scrreprt}
\usepackage[a4paper, right=20mm, left=20mm,top=20mm, bottom=30mm, marginparsep=5mm, marginparwidth=5mm, headheight=7mm, headsep=15mm,footskip=15mm]{geometry}

%Papierausrichtungen
\usepackage{pdflscape}
\usepackage{lscape}

%Deutsche Umlaute, Schriftart, Deutsche Bezeichnungen
\usepackage[utf8]{inputenc}
\usepackage[T1]{fontenc}
\usepackage[ngerman]{babel}

%quellcode
\usepackage{listings}

%tabellen
\usepackage{tabularx}

%listen und aufzählungen
\usepackage{paralist}

%farben
\usepackage[svgnames,table,hyperref]{xcolor}

%symbole
\usepackage{latexsym,textcomp}

%font
\usepackage{helvet}
\renewcommand{\familydefault}{\sfdefault}

%Abkürzungsverzeichnisse
\usepackage[printonlyused]{acronym}

%Bilder
\usepackage{graphicx} %Bilder
\usepackage{float}	  %"Floating" Objects, Bilder, Tabellen...
\usepackage[space]{grffile} %Leerzechen Problem bei includegraphics
\usepackage{wallpaper} %Seitenhintergrund setzen
\usepackage{transparent} %Transparenz

%for
\usepackage{forloop}
\usepackage{ifthen}

%Dokumenteigenschaften
\title{Repetitionsfragen InfSi2}
\author{Tobias Blaser}
\date{\today{}, Rapperswil}


%Kopf- /Fusszeile
\usepackage{fancyhdr}
\usepackage{lastpage}

\pagestyle{fancy}
	\fancyhf{} %alle Kopf- und Fußzeilenfelder bereinigen
	\renewcommand{\headrulewidth}{0pt} %obere Trennlinie
	\fancyfoot[L]{Seite \thepage/\pageref{LastPage}} %Fusszeile mitte
	\fancyfoot[R]{\today{}} %Fusszeile rechts
	\renewcommand{\footrulewidth}{0.4pt} %untere Trennlinie

%Kopf-/ Fusszeile auf chapter page
\fancypagestyle{plain} {
	\fancyhf{} %alle Kopf- und Fußzeilenfelder bereinigen
	\renewcommand{\headrulewidth}{0pt} %obere Trennlinie
	\fancyfoot[L]{Seite \thepage/\pageref{LastPage}} %Fusszeile mitte
	\fancyfoot[R]{\today{}} %Fusszeile rechts
	\renewcommand{\footrulewidth}{0.4pt} %untere Trennlinie
}

\usepackage{changepage}

% Abkürzungen für Kapitel, Titel und Listen
\input{commands/shortcutsListAndChapter}
\input{commands/TextStructuringBoxes}

%links, verlinktes Inhaltsverzeichnis, PDF Inhaltsverzeichnis
\usepackage[bookmarks=true,
bookmarksopen=true,
bookmarksnumbered=true,
breaklinks=true,
colorlinks=true,
linkcolor=black,
anchorcolor=black,
citecolor=black,
filecolor=black,
menucolor=black,
pagecolor=black,
urlcolor=black
]{hyperref} % Paket muss unbedingt als letzes eingebunden werden!

\usepackage{graphicx}
\begin{document}

% Inhaltsverzeichnis
\tableofcontents
\clearpage

\ch{Kryptographische stärke}
\ol
	\li Abb. \ref{csnt}: Füllen Sie die Tabelle aus.
	\img{img/r1.1.jpg}{}{1}{csnt}
	\li Was ist das Ziel der NSA Suite? Wozu dienen Suite A und B?
	\li Erklären Sie wie das Prinzip von Elliptic Curves funktioniert.
	\li Wie wird mit EC ``Secret Key Exchange'' gemacht?
	\li Was sind ECDH, ECIES und ECDSA?
	\li Was ist AEAD, welchen Vorteil bringt es?
\olS


\ch{Physical Layer Security}
\olR
	\li Erklären Sie die beiden veralteten Sicherheitsverfahren: Voice Scrambling, Frequency Hopping
	\li Wie funktioniert Key Distribution mittels Quanten?
	\li Wie kann bei Quanten festgestellt werden, wenn jemand mithört?
	\li Wie funktioniert key Distribution mit Photonen nach BB84? Wie wird hier verhindert, das jemand mithört? Welche praktischen Einschränkungen besitzt die Technologie?
	\li Welche technischen Schwierigkeiten gibt es heute grundsätzlich mit Photonendatenübertragung?
	\li Warum braucht ein Sicherheitssystem mit Photonen/Quanten immer zwei Kommunikationskanäle?
\olS


\ch{Schlüsselmaterial und Zufallszahlen}
\olR
	\li Erklären Sie, wie HMAC RFC 2104 funktioniert.
	\li Erklären Sie, was PRF ist, wie es funktioniert und warum er generierte Key Stream keine grössere Entropie aufweisen kann als das ursprüngliche Secret.
	\li Warum wird die Berechnung des TLS 1.1 Master secrets auf zwei Teile (MD5/SHA-1) aufgeteilt? Wie wird bei TLS 1.1 Schlüsselmaterial generiert'
	\li Welche Möglichkeiten gibt es, um richtige Zufallszahlen zu generieren?
	\li Was sind Hardware basierte Zufallszahlengeneratoren?
	\li Wie funktioniert der Quantum Random Number Generator von IDQ?
	\li Wie können Sie die statistische Verteilung von Zufallsmaterial verändern (von Neumann)?
\olS


\ch{Data Link Layer Security}
\olR
	\li Erklären Sie wie die Authentisierung bei EAP mit Radius Server funktioniert.
	\li Was ist die Secure Device ID (DecID)? Wozu kann sie verwendet werden? Können Sie die DevID verändern? begründen Sie. Welche weiteren Informationen / Elemente sollte das DevID Modul noch beinhalten?
	\li Was ist eine Connectivity Association? Welchen Vorteil bieten Secure Channel und Secure Association?
	\li Wie ist das MACsec Frame Format aufgebaut? Erklären Sie die Felder SecTag, Secure Data, ICV.
	\li Wie funktioniert die Key Distribution mit MKA, SAK und CAK?
	\li Wie können Sie mit EAP dynamische CAK's ermöglichen?
\olS


\ch{Application Security}
\olR
	\li Was ist das OWASP?
	\li Nennen Sie die 5-A Prinzipien, die bei Web Application Security zum Tragen kommen.
	\li Erklären Sie, wie eine 3-Tier Web Application aufgebaut ist und welche Security Massnahmen sie auf jedem Tier vornehmen müssen.
	\li Was ist Clickjacking?
	\li Was ist Cross-Site Scripting? Erklären Sie die drei Arten.
	\li Fassen sie kurz den Whitehat Security Report zusammen.
	\li Notieren Sie typische CIA-Sicherheitsprobleme bei Webanwendungen (Chat Vorlesung).
	\li Was sind die OWASP Top 10. Nach welchen Kriterien werden sie bestimmt? Nenen Sie die ersten fünf.
	\li Erklären Sie das bereits aus InfSi1 bekannte OWASP Thread Model.
	\li Was sind Vulnerability Factors? Nennen Sie einige.
	\li Nennen Sie die fünf Phasen des OWASP Security Testing Guide.
	\li Nennen Sie die wichtigsten Punkte aus dem OWASP Application Security Architecture Cheat Sheet.
	\li Erklären Sie den Thread: Insufficient TLS Protection
\olS


\se{Web Application Security}
\olp{
	\li Aus welchem Grund richtet sich heute ein Grossteil der Angriffe gegen Applikationen und nicht die unteren Schichten?
	\li Wie funktioniert eine SQL injection? 
	\li Nennen Sie fünf Massnahmen gegen SQL Injection. Welche ist wie effizient?
	\li Was ist eine Bypass Authentication?
	\li Warum sollte im Frontent nie eine aufschlussreiche Fehlermeldung ausgegeben werden? Wie kann ein Angreifer z.B. bei einem username/password Formular auch ohne aufschlussreiche Fehlermeldung Informationen darüber erhalten, ob das Passwort oder der Username erhalten?
	\li Welche Angriffsmethode wird gegen Error Delay (Bei einer Falscheingabe wird der Zugang für eine bestimmte Zeit gesperrt) angewandt? 
	\li Erklären Sie den Unterschied zwischen Authentisierung und Authorisierung.
	\li Welches sind die drei Authentisierungsfaktoren?
	\li Welche Probleme birgt Form Autocompleten des Browsers? Was kann man als Webentwickler dagegen tun?
}

\se{Data Leak Protection}
\olp{
	\li Warum sollen Daten geschützt werden?
	\li Wo in einer Unternehmung sind Unternehmensdaten gespeichert?
	\li Wo und wie können Daten gestohlen geleakt werden?
	\li Was ist Egress Control und was Usage Control? Erklären Sie, wie beide Konzepte versuchen Daten zu schützen. Wie können Sie die beiden Ansätze kombinieren?
	\li Erklären Sie, welche Stufen von Anonymen Testdaten es gibt, und wie Testdaten ein Sicherheitsrisiko darstellen können.
	\li Was ist IRM? Wie unterscheidet es sich vom DRM?
	\li Was ist das ``AD RMS''?
	\li Wie werden beim AD RMS Dokumente geschützt? Können Sie ein geschütztes Dokument auch du Hause öffnen?
	\li Warum funktionieren Systeme wie AD RMS nur in einer klar definierten und vorgegebenen Umgebung? Können Sie AD RMS mit BYOD kombinieren?
	\li Wie beurteilen Sie grundsätzlich die Situation von BYOD in Unternehmungen?
	\li Welche Probleme können Sie mit IRM nicht lösen?
	\li White Paper - Enterprise Security:
		\oli{
			\li Wie vertrauenswürdig ist das Paper? Autor?
			\li\label{dataLeaksReasons} Welche drei Ursachen sind gemäss Paper hauptverantwortlich für Data Leaks? (siehe Anhang)
		}
}


\ch{Anonymität}
\olp{
	\li Aus welchem Grund ist die Möglichkeit zur Anonymität notwendig?
	\li Wie funktioniert ein Remailer? Wie anonym ist der entsprechende User damit?
	\li Wo liegt der Schwachpunkt des Remailers?
}
\se{Mix Net}
\olp{
	\li Erklären Sie ``David Chaum's Cascade of Mixes''.
	\li Wie wird verhindert, dass die Knotenserver die andern Server kennen?
	\li Was passiert, wenn ein Server unterwandert wurde? Was wenn der Exit Knoten unterwandert wurde?
	\li Warum sollten die verschiedenen Knoten möglichst in unterschiedlichen Ländern stehen?
	\li Welche Aufgaben übernimmt ein Mix Knoten?
	\li Erklären Sie den Unterschied zwischen High-Latency und Low-Latency Anonymisierungsdiensten in Bezug auf anonymität und Geschwindigkeit.
	\li Wo liegt der Schwachpunkt des Mix Net?
}
\se{Tor}
\olp{
	\li Was ist Tor?
	\li Wie ist das Datenformat bei Tor aufgebaut?
	\li Wie ordnet Tor einer ausgehenden Nachricht bei der entsprechenden Antwort den Empfänger zu?
	\li Skizzieren Sie mit einem Diagramm, wie ein Tor Circuit aufgebaut wird? Zeichnen Sie auch die Verschlüsselung ein.
	\li Wie können Sie mit Tor einen Service anonymisieren? Was ist ein Rendevouz Point?
	\li Sie benutzen das Tor bundle und browser mit dem mitgelieferten Tor Browser (modifizierten Firefox). Trotzdem können Sie Webseiten mit einer bestimmten Sicherheit bei einem erneuten Besuch zu ihrem vorherigen Besuch zuordnen. Wie geht das? Was können Sie dagegen tun?
}


\ch{VPN}
\se{Point to Point}
\olp{
	\li Erklären Sie, wie PPP funktioniert. Wie sicher ist PPP?
	\li Wie ist das Encrypted Control Protocol ECP aufgebaut?
	\li Welche Authentifizierungsmöglichkeiten gibt es bei PPP?
}

\se{Layer 2/3/4 VPN}
\olp{
	\li Wie funktioniert das Layer2 Tunneling Protocol L2TP?
	\li Welche Tunneling Modes gibt es? Wie sicher ist L2TP?
	\li Wie funktioniert L2TP über IPsec?
	\li Erklären Sie, wie TLS based Tunneling aufgebaut ist.
	\li Stellen Sie L2TP, IPsec und TLS Tunnel einander gegenüber und nennen Sie je Vor- und Nachteile.
}

\se{MPLS}
\olp{
	\li Was ist MPLS? Wozu dient es? Wie ist der MPLS Layer2 Shim Header aufgebaut, wo wird er eingefügt?
}

\se{IPsec Transport Mode}
\olp{
	\li Wie funktioniert der IPsec Transport Mode? Warum sollten die IPsec Pakete sowohl verschlüsselt wie authentifiziert sein?
	\li Erklären Sie den Aufbau des IPsec Authentication Header AH.
}

\se{IPsec Tunnel Mode}
\olp{
	\li Wie funktioniert der IPsec Tunnel Mode? Welchen Vorteil bietet er? Wie sehen die Pakete aus, welcher Teil ist verschlüsselt, welcher Authentifiziert? Was ist der Trailer?
	\li Wie funktioniert die Authentication with Associated Data AEAD? Wie viel Oberhead entsteht dadurch?
}


\se{Internet Key Exchange}
\olp{
	\li Was ist die Aufgabe des IKE?
	\li Was ist eine Security Association?
}

\sse{IKEv1}
\olp{
	\li Skizzieren Sie, wie IKEv1 im Main Mode funktioniert. Skizzieren Sie was für Pakete mit welchen Inhalten übertragen werden, und welche Teile der Pakete verschlüsselt sind.
	\li Wie funktioniert IKEv1 im Main Mode mit Pre-Shared Keys? Warum wird dieser Modusin der Praxis ncht eingesetzt?
	\li Wie funktioniert der IKEv1 im Aggressive Mode? Wo liegt die Schwachstelle?
	\li Wie funktioniert ein Man-in-the-Middle Attack auf IKEv1 im Aggressive Mode?
	\li Wie funktioniert IKEv1 Phase2
}

\sse{IKEv2}
\olp{
	\li Nennen Sie einige entscheidende Neuerungen / Änderungen bei IKEv2
	\li Skizzieren Sie auf, wie Authentication und die etablierung einer ersten Child SA funktioniert. Skizzieren Sie, was für Pakete mit welchen Inhalten übertragen werden, und welche Teile verschlüsselt sind.
	\li Wie funktioniert der Cookie Mechanismus?
	\li Wie funktioniert das hinzufügen einer zweiten Child SA?
	\li Beschreiben Sie die beiden Anwendungsfälle ``VPN zwischen Standorten'' und ``Remote Access''.
	\li Nennen Sie zwei möglichkeiten, die Authentication an einen andern Dienst auszulagern. Welche Variante ist zu bevorzugen? Warum?  
	\li Was müssen Sie machen, damit Sie mit IKE und VPN durch NAT Router durchkommen? Skizzieren Sie die entsprechenden Pakete.
}


\ch{DNSSEC}
\olp{
	\li Wie funktioniert die Kaminski Attacke auf das DNS System?
	\li Erklären Sie, wie ein DNS Request und eine DNS Response aufgebaut sind.
	\li Wie funktioniert die DNSSEC Chain of Trust? Was sind KSK und ZSK?
	\li Was ist die RRSIG?
	\li Was ist DANE? Wozu dient es?
	\li Wie können Sie mit DANE Server- und CA Zertifikat Verifizieren?
	\li Wie können Sie mit DANE das CA Zertifikat oder den Public Key erhalten?
	\li Wie funktionieren Self-Signed Server Zertifikate mit DANE?
	\li Wie funktioniert das verifizieren von rohen RSA Keys mit DANE?
	\li Wie können Sie mit DANE das Server Zertifikat oder den Public Key erhalten?
	\li Wie funktioniert der DNS Root Zone Singing Process?
	\li Welche Möglichkeiten gibt es, falls ein Key eines DNSSEC Servers kompromitiert wird? Kann der Key zurückgezogen werden?
	\li Wer betreibt die Root DNSSEC Server? Wie wird die Echtheit des Root KSK/ZSK garantiert?
}


\ch{NAC}
\se{Firewall}
\olp{
	\li Erklären Sie das Grundsätzliche Prinzip von Firewalls
	\li Was ist ``Statefull Inspection''?
}

\se{NAC}
\olp{
	\li Auf welche Arten wird bei NAC die User authentication realisiert?
	\li Wozu wird ein ``Configuration Assessment'' durchgeführt?
	\li Erklären Sie die drei NAC Zustände/Zonen
	\li Wie funktioniert TNC? Skizzieren Sie auf, was auf NAC Layer ausgetauscht wird.
	\li Wie wird bei NAC eine sichere Verbindung zum Austausch der Pakete für den Identity Check und den Health Check aufgebaut?
	\li Was ist Network Endpoint Assessment?
	\li Was ist ein ``Metadata Access Point''?
	\li Wo liegt der Vorteil von einem zentralen Map Service? Wie passt der Service in die NAC Architektur?
	\li Kann NAC erkennen/schützen, wenn ein ``Lying Endpoints'' vorliegt?
}


\ch{Buffer Overflow}
\olp{
	\li Erklären Sie, wie mittels Buffer Overflow fremder Code ins System eingeschleusst werden und ausgeführt werden kann.
	\li Welche modernen Schutzmassnahmen verringern die Gefahr von durch Buffer Overflow eingeschleusstem Code? Nennen Sie jeweils auch, wie wirksam die Schutzmassnahme ist.
}


\ch{Smart Cards}
\olp{
	\li Was ist eine SmartCard? Wozu wird Sie verwendet?
	\li Nennen Sie die Eigenschaften der folgenden Kartentypen:
		\oli{
			\li Memory Card
			\li USB Token
			\li SIM Card
			\li Crypto Card
			\li Java Card
		}
	\li Was sind proximity und Vicinity Cards? Warum diese Cards nur mit kleinen Funkdistanzen gebaut werden?
	\li Was sind Display Cards?
	\li Was ist NFC? Erklären Sie Secure NFC.
	\li Abb. \ref{smCdElco}: Erklären Sie die eletrischen Kontakte:
		\img{img/v13.6.jpg}{Elektrische Kontakte Smart Cards}{0.5}{smCdElco}
	\li Mit welchen Massnahmen wird die Karte Physikalisch abgesichert?
	\li Wie ist das SmartCard File System aufgebaut? Wie sind Files intern aufgebaut? Welche Filetypen gibt es?
	\li Welche Arten von SmartCard Messages gibt es? Wozu werden Sie eingesetzt? Erklären Sie deren Aufbau.
	\li Wie kann mit PC/SC oder PKCS auf die Card zugegriffen werden?
	\li Wie ist das PKCS\#15 Cryptographic Token Information Format aufgebaut?
	\li Welche Systeme gibt es, um mit SmartCards zu bezahlen?
	\li Wie können SmartCards als Authentication verwendet werden?
}


\ch{TPM}
\olp{
	\li Was ist ein TPM? Wozu wird es gebraucht?
	\li Was beinhaltet ein TPM alles?
	\li Warum ist es nicht möglich, mittels TPM Festplatten zu ver- und Entschlüsseln? Wie kam es dazu, dass dies hardwaretechnisch nicht besser gelöst wurde?
	\li Wozu dient der Storage Root Key?
	\li Erklären Sie StorK, AIK, SigK, BindK, MigrK, SymK. Dem MigrK kommt insbesondere eine hohe Bedeutung zu. Was passiert, wenn Ihnen ihr Notebook defekt ist, und sie keinen MigrK generiert haben?
	\li Skizzieren Sie auf, wie hybride File encrytion mittels Storage Key funktioniert. Was passiert, wenn Sie den Storage Key reseten?
	\li Erklären Sie, was Bindung und Sealing bedeuten, und welchen Einfluss sie haben.
	\li Was is ``Static Root of Trust for Measurement (SRTM)''?
	\li Warum ist es nicht möglich, mit TPM das komplette System zu schützen?
	\li Wozu dienen die PCR Register im TPM? Wie kann damit die Gesundheit des Systems gewährleistet werden? Wie funktioniert dies genau?
	\li Nennen Sie einige Anwendungsmöglichkeiten von TPM
}




\appendix
\ch{Antworten}
\ref{dataLeaksReasons}: well meaning insiders, target attacks, malcious insiders


\end{document}
