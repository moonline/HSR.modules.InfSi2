%Pakete;
%A4, Report, 12pt
\documentclass[ngerman,a4paper,12pt]{scrreprt}
\usepackage[a4paper, right=20mm, left=20mm,top=30mm, bottom=30mm, marginparsep=5mm, marginparwidth=5mm, headheight=7mm, headsep=15mm,footskip=15mm]{geometry}

%Papierausrichtungen
\usepackage{pdflscape}
\usepackage{lscape}

%Deutsche Umlaute, Schriftart, Deutsche Bezeichnungen
\usepackage[utf8]{inputenc}
\usepackage[T1]{fontenc}
\usepackage[ngerman]{babel}

%quellcode
\usepackage{listings}

%tabellen
\usepackage{tabularx}

%listen und aufzählungen
\usepackage{paralist}

%farben
\usepackage[svgnames,table,hyperref]{xcolor}

%symbole
\usepackage{latexsym,textcomp}
\usepackage{amssymb}

%font
\usepackage{helvet}
\renewcommand{\familydefault}{\sfdefault}

%durch- und unterstreichen
\usepackage{ulem}

%Abkürzungsverzeichnisse
\usepackage[printonlyused]{acronym}

%Bilder
\usepackage{graphicx} %Bilder
\usepackage{float}	  %"Floating" Objects, Bilder, Tabellen...
\usepackage[space]{grffile} %Leerzechen Problem bei includegraphics
\usepackage{wallpaper} %Seitenhintergrund setzen
\usepackage{transparent} %Transparenz

%Tikz, Mindmaps, Trees
\usepackage{tikz}
\usetikzlibrary{mindmap,trees}
\usepackage{verbatim}

%for
\usepackage{forloop}
\usepackage{ifthen}

%Dokumenteigenschaften
\title{Summary InfSi2}
\author{Tobias Blaser}
\date{\today{}, Uster}


%Kopf- /Fusszeile
\usepackage{fancyhdr}
\usepackage{lastpage}

\pagestyle{fancy}
	\fancyhf{} %alle Kopf- und Fußzeilenfelder bereinigen
	\renewcommand{\headrulewidth}{0pt} %obere Trennlinie
	\fancyfoot[L]{\jobname} %Fusszeile links
	\fancyfoot[C]{Seite \thepage/\pageref{LastPage}} %Fusszeile mitte
	\fancyfoot[R]{\today{}} %Fusszeile rechts
	\renewcommand{\footrulewidth}{0.4pt} %untere Trennlinie

%Kopf-/ Fusszeile auf chapter page
\fancypagestyle{plain} {
	\fancyhf{} %alle Kopf- und Fußzeilenfelder bereinigen
	\renewcommand{\headrulewidth}{0pt} %obere Trennlinie
	\fancyfoot[L]{\jobname} %Fusszeile links
	\fancyfoot[C]{Seite \thepage/\pageref{LastPage}} %Fusszeile mitte
	\fancyfoot[R]{\today{}} %Fusszeile rechts
	\renewcommand{\footrulewidth}{0.4pt} %untere Trennlinie
}

\usepackage{changepage}

% Abkürzungen für Kapitel, Titel und Listen
\input{commands/shortcutsListAndChapter}
\input{commands/TextStructuringBoxes}

%links, verlinktes Inhaltsverzeichnis, PDF Inhaltsverzeichnis
\usepackage[bookmarks=true,
bookmarksopen=true,
bookmarksnumbered=true,
breaklinks=true,
colorlinks=true,
linkcolor=black,
anchorcolor=black,
citecolor=black,
filecolor=black,
menucolor=black,
pagecolor=black,
urlcolor=black
]{hyperref} % Paket muss unbedingt als letzes eingebunden werden!

\usepackage{graphicx}
\begin{document}

% Inhaltsverzeichnis
\tableofcontents
\clearpage

\ch{Repetition}

\se{Kryptographische stärke im 2013}
\exam{Schlüsselstärken wissen an Prüfung}

\ul
	\li Symetrische Verschlüsselung: 
		\ul
			\li AES \textbf{128bit}, 192bit, \textbf{256bit} \\
			\ra Es gibt keine Shortcut Attacken bei 128, nur Brute Force. 256 und 192 haben jeweils eine etwas kleinere Stärke (ca 220, 180) weil man durch das verwendete Gleichungssystem etwas effizienter als mit Brute Force angreifen kann.
			\li Twofish 
			\li 3DES 168bit (3x56bit) \ra Stärke 112bit (doppelte DES Veschlüsselung) \\
				\ra Es gibt eine Attacke von beiden Enden her, darum braucht 3DES in der Mitte noch die 3. DES Verschlüsselung und kommt auf eine Stärke von doppeltem DES
			\li Camellio
		\ulE
	\li Data Integrity (Hash): 
		\ul
			\li SHA-3 (Keccak) 224, 256, 384, 512bit \\
				\ra auf 64Bit Rechnern ist 512bit schneller als 256bit
			\li \textbf{SHA-2 (NIST)}
		\ulE
	\li Key Exchange (between peers):
		\ul
			\li DH \textbf{2048bit}, 1576bit \\
			\ra Aufwand wächst gegenüber symetrischen Verfahren mit der 3.Potenz des Modulus der Primzahl, daher Key viel länger
		\ulE
	\li Digital Signature: 
		\ul
			\li RSA \textbf{2048bit}, 4096bit \\
			\ra Root Server, die ein Zertifikat für 20 Jahre lösen, verwenden schon heute 4096
			\li DSA (Digital Signature Algorithm, erlaubt nur zu verschlüsseln nicht zu signieren) \\
			\ra mit eliptischen Kruven: ECDSA
		\ulE
	\li Public Key:
		\ul
			\li RSA 2048bit, 4096bit für langjährige Ausrichtung
			\li El Gamal (verwandt mit DH) 2048bit
		\ulE
	\li User Password: 22 Zeichen \ra 128bit, 12 absolutes Minimum
\ulE

\se{NSA Suites}
\img{img/v1.1.jpg}{NSA Suite B 128bit security (CONFIDENTIAL)}{0.75}{}
\img{img/v1.2.jpg}{NSA Suite B 192bit security (SECRET)}{0.75}{}

Alle Verfahren sollen gleich stark sein, damit es kein schwächstes Verfahren in der Kette gibt.

\ch{Elliptische Kurven}
\img{img/v1.3.jpg}{}{0.75}{}
\ul
	\li Polynom 3. Grades
	\li EC sind immer spiegelsymetrisch zur X-Achse, weil Quadratwurzeln immer zwei Lösungen besitzen
	\li Gute elliptische Kurve: Keine Nullstellen fallen zusammen, x\^3 besitzt 3 unterschiedliche Nullstellen.
	\li Operationen müssen Assoziativ sein
	\li Oft auch Kommutativ
\ulE

\expl{Algebraische Gruppe}{Zahlensystem, in dem bestimmte Regeln gelten. 
	\ul
		\li Wendet man auf Gruppenmitglieder Operationen an, so liegt das Resultat wieder innerhalb der Gruppe
		\li Jede Gruppe besitzt ein neutrales Element e
		\li Jede Gruppe besitzt ein inverses Element a
	\ulE}
\img{img/v1.4.jpg}{}{0.75}{}
\img{img/v1.5.jpg}{Punkte auf einer elliptischen Kurve, die eine Gruppe bilden}{0.75}{}
\ul
	\li Alle Elemente, die auf einer elliptischen Kruve liegen, bilden eine algebraische Gruppe.
	\li Gearbeitet wird mit Punktaddition:
		\ul
			\li Punkte werden mti Linie verbunden \ra neuer Punkt auf Line \ra spiegelung davon is Resultat
		\ulE
	\li Der Punkt im Unendlichen ist das neutrale Element
	\li P + das neutrale Element ergibt wieder P
	\li Um wie mit DH damit arbeiten zu können, muss ich einen Punkt mehrmals zu sich selbst addieren
\ulE
\img{img/v1.6.jpg}{Neutrales und inverses Element}{0.75}{}
\img{img/v1.7.jpg}{Punkt duplikation - einen Punkt zu sich selbst addieren}{0.75}{}
\img{img/v1.8.jpg}{Punkt Iteration, Addieren eines Punktes k-1 mal zu sich selbst}{0.75}{}
\img{img/v1.9.jpg}{}{0.75}{}

\se{Praktische Anwendung}
\img{img/v1.10.jpg}{}{0.75}{}
\ul
	\li Kurve wird bekannt gegeben: Primzahl b, Parameter a,b
	\li Startpunkt P auf Kurve wird auch bekannt gegeben
	\li Bob und Alis wählen eine Zahl a, die bestimmt, wie viel mal P mit sich selbst addiert wird \ra Bildpunkt 
	\li Bildpunkt wird öffentlich übertragen
	\li aus den übertragenene Elementen kann nur mit sehr grossem Aufwand a berechnet werden
	\li das Wachstum der geheimnisse a und b korreliert fast linear mit der Verschlüsselungsstärke
\ulE

\se{AEAD}
\ul
	\li Verschlüsselung + Hash in einem Schritt
	\li Gegenüber einzelner Verschlüsselung und einzelner Signaturerstellung viel schneller
	\li Kaum langsamer gegenüber nur Verschlüsselung
\ulE


\ch{Physical Layer Security}
Früher Verwendet:
\ul
	\li Voice Scrambling (Time Multiplex)
	\li Frequency Hopping
\ulE

\se{Quantenkryptographie}
Splitten von Photonen in zwei Photenen, die verschränkt sind.
\img{img/v2.1.jpg}{}{0.75}{}
\expl{Polarisiertes Licht}{Nach dem horizontalen oder vertikalen elektrischen Feld ausgerichtetes Licht}
\ul
	\li Nicht gemessene Photonen haben beide polarisierten Zustände
	\li Zwei Set von Filter: 0°, 90° (Sieht aus wie ein Schweizerkreuz) und 45°, 115°.
	\li Messen Empfänger und Sender gleich (polarisiert), so haben die beiden Photonen den gleichen Zustand.
	\li 2. Zufallsmessung
	\li Horcht jemand mit, so verändert er damit den Quantenzustand und die beiden Kommunikationsteilnehmer bemerken dies. \ra Bob erhält kein Photon
	\li Klaut Eve das Photon und speist ein anderes ein, so funktioniert dies nicht, weil verschränkte Photonen nicht dupliziert werden können.
	\li Diese Quanten laufen über eine sepparate Glasphaser
	\li Kommunikationsparter teil sich nicht mit, was sie gemessen haben, sondern mit welchen Filterset. Haben SIe mit unterschiedlichen Filtersets gemessen, so werden diese Photonen verworfen. \ra 50\% Ausschuss
	\li Wegen diesen 50\% Verwurf kann der Quantenkanal nicht für Streamcipher verwendet werden. Zudem ist der Schlüsselstrom ziemlich niederratig. \ra Quantenkanal wird nur für Session-Schlüsselaustausch verwendet.
	\li Problem mit Distanz: Repeathen nicht möglich.
\ulE

\se{BB84 Key Distribution System}
\ul
	\li Alice erzeugt normalePhotonen und sendet diese an Bob
	\li Photon wird moduliert
	\li Bob wählt zufällig Filterset
	\li Wenn Eve ein Photon klaut, kriegt Bob keines
	\li Nach gewisser Anzahl Photonen tauschen Bob und Alice das verwendete Filterset aus und Alice gibt die Modulation bekannt (orthogonal / 45° verschoben). \ra Nicht passende Filtersets werden verworfen
	\li Speist Eve zusätzliche Photonen ein, so kann dies detektiert werden
	\li Es ist praktisch unmöglich einen Laser zu bauen, der nur ein Photon ausgibt
	\li Problematisch, wenn der Laser mehr als ein Photon ausgibt, weil Eve dann eines der Photonen klaut. \ra Eve klaut natürlich immer eines, daher erhält Bob nur die Photonen, die mehrfach ausgesendet werden.
	\li Dämpfung in der Glasfaser: bei 10dB Dämpfung kommen nur noch 10\% an, bei 20db nur 1\%.
	\li Maximale Distanz: 1000km \ra 30dB Dämpfung
	\li Daher ist die Netodatenrate sehr tief und es fällt gar nicht auf, wenn Eve Photonen klaut.
\ulE

\img{img/v2.2.jpg}{}{0.75}{}

\sse{Erhältliche Systeme}
\img{img/v2.3.jpg}{Layer 2 Encription with Quantum Key Distribution}{0.75}{}
\ul
	\li Untersten Kanal ist Key Distribution Kanal (1Key/min)
	\li Basierend auf RR84 und SARG, bis 50km Distanz zwischen Ersatz Rechenzentrum
	\li 10Gbit/s mit AES-256
\ulE
\img{img/v2.4.jpg}{}{0.75}{}


\se{Schlüssel Material und Randoom Nummern}
\img{img/v2.5.jpg}{HMAC}{0.75}{}
\ul
	\li Dokument wird zusammen mit Key gehashed und das ganze erneut mit Outerkey zu MAC verarbeitet.
\ulE
\img{img/v2.6.jpg}{PRF pseudo randoom function}{0.75}{}
\ul
	\li Seed wird gehashed mit Secret zusammen
	\li Autput wird erneut als Input für Hash verwendet
	\li So wird Stück für Stück genügend Zufallsmaterial generiert
	\li Zur Sicherheit wird jeder Output am Schluss erneut gehashed, damit nicht auf den vorherigen Output zurückgeschlossen werden kann
	\li Entropie des keys (Secret) wird ``verdünnt'' auf den ganzen Bitstream \ra nimmt nicht zu
\ulE
\exam{PRF Funktion nicht genau kennen, nur wissen dass es ein Verdünnen von Schlüsselmaterial ist}

\sse{TLS1.1 Master secret}
\img{img/v2.7.jpg}{TLS 1.1 Master Secret Berechnung}{0.75}{}
\ul
	\li Premastersecret (echte Entropie) wird aufgeteilt und erste Hälfte durch PRF-MD5 durchgelassen, zweite Hälfte durch PRF-SHA-1 durchgelassen. \\
	 \li Grund dafür: Angst vor Fall von MD5, Angst vor Backdoor in SHA-1
	 \li Beide Teile werden anschliessend verknüpft
\ulE
\img{img/v2.8.jpg}{Generating TLS 1.1 Key Material}{0.75}{}


\se{Generieren von wahren randoom Numbers}
\ul
	\li Zeicheneingabe von Tastatur besitzt wenig Entropie \ra Verwendet wird die Zeit zwischen den Tastenschlägen
	\li Mausbewegung
	\li Rauschen aus der Soundkarte
	\li Zugriffszeiten auf die Harddisk (Schwankungen durch Luftturbulenzen in echanischen HD's)
	\li Am Besten möglichst viele Quellen zusammenhashen
\ulE
\important{Hashen erhöht die Entropie nicht! Verbessert nur die Statistik}
\expl{Entropie Generator Ausfall}{Muss detektiert werden, weil sonst ein Strom mit null Entropie empfangen wird.}

\sse{Hardware basierte wahre Random Generatoren}
\ul
	\li Jeder Halbleiter rauscht \ra Auf Prozessor wird Diode gesetzt, die durch Rauschen Zufallszahlen erzeugt
	\li Thermal Nose Sources
\ulE

\img{img/v2.9.jpg}{Photonengenerator, dessen Strom durch Spiegel umgelenkt werden}{0.75}{}
\img{img/v2.10.jpg}{}{0.75}{}

\sse{Randomness überprüfen und korrigieren}
\ul
	\li Problem: zu viele Nullen \ra Bits werden nach Auftrittswahrscheinlichkeit substituiert.
\ulE

\ch{Data Link Layer Security}
\se{EAP}
\img{img/v3.1.jpg}{Gerät ist entweder Suplicant oder Authentiater (Antragsteller oder Authentifizierer)}{0.75}{}

\definition{EAPOL}{EAP over LAN}
Gewünscht: Verschlüsseln der Trunkleitung zum Radiusserver

\se{IEEE 802.1AR -DevID}
\definition{DevID}{Secure Device ID}
\ul
	\li Jedes Gerät wird ab Werk mit Schlüssel für EC, RSA ausgeliefert
	\li In einem Rom gespeichert, kann nicht gelöscht werden
	\li Bereits ein Zertifikat gespeichert, welches vom Hardwarehersteller signiert wurde
	\li Lokaler Administrator kann beliebig viele Device ID's hinzufügens
	\li Befindet sich in geschütztem Chip
\ulE
\img{img/v3.2.jpg}{}{0.75}{}
\ul
	\li SHA 256 vorgeschrieben
	\li ANwendungen: EAP-TLS Authentisierung (Nicht User ID sondern Device ID)
	\li Mit TPM haben die meissten PC's bereits heute Möglichkeiten für eine DeviceID
\ulE

\se{MAC Layer Security}
\ul
	\li Verschlüsselung muss effizient sein wegen häufigem rekeying
	\li Bestimmte Benutzer werden zu Gruppen zusammengefast (Connectivity Association)
\ulE
\definition{CAK}{Connectivity Authentication Key}
\img{img/v3.3.jpg}{CA}{0.5}{}
\img{img/v3.4.jpg}{Secure Channel für jede Verbindung zwischen den Teilnehmern}{0.5}{}
\ul
	\li Jeder Channel besitzt einen eigenen Schlüssel (Channel sind gerichtet)
	\li Schlüsselwechsel muss fliegend geschehen können: Pakete werden mit altem Schlüssel verschlüsselt und irgendwann wird gewechselt
	\li Jedes Paket enthält Channel und welcher Schlüssel zum Entschlüsseln verwendet werden soll
\ulE
\img{img/v3.5.jpg}{Association Number legt fest, ob alter oder neuer Key verwendet wurde}{0.75}{}

\sse{Sonderfall Peer-Peer, Peer-AP}
\img{img/v3.6.jpg}{}{0.75}{}

\sse{MAC sec}
\img{img/v3.7.jpg}{MAC sec Frame Format}{0.75}{}
\ul
	\li Security Tag: 8Byte oder 16Byte
	\li VErschlüsselung: Paket wird mit AES im Countermode verschlüsselt
	\li ICV ist MAC, krypt. Checksumme (16Byte)
	\li Ethernet Frame wird gesprengt wie bei getaggten VLAN Paketen
	\li FCS: Checksumme auf Bitfehler
\ulE

\img{img/v3.8.jpg}{SecTag Format}{0.75}{}
\ul
	\li Ethertype für MACSec (> 1500)
	\li im TCI ist definiert, ob SCI (Secure Channel Identifiert) mitgeschickt wird \\
	\ra Wenn SCI aus der MAC Adresse berechnet werden kann, kann er weggelassen werden
	\li Hash von SCI wird übertragen, auch wenn SCI nicht übertragen wird
	\li Wenn Paket zu kurz für Kollisionsdetektion, wird Tagging NACH dem ICV eingefügt \\
		\ra Problem bei Ethernet ohne Längenfeld: Wo beginnt das Padding? Chechsumme falsch \ra Secure TAG enthält Länge im SL (Short Length) Field (Feld gibt an, wenn Paket kürzer als 64Byte, wie lange es ist)
	\li schutz gegen Replay Attacken: PN Feld enthält Paket Number
\ulE

\img{img/v3.9.jpg}{TCI - TAG Control Information}{0.75}{}
\ul
	\li Wenn Bit E gesetzt: Verschlüsselung, sonst nur Checksum
\ulE

\img{img/v3.10.jpg}{Verschlüsselung im AES Counter Mode, SAK und CAK bieten 128bit \ra durchgehend gleiche Verschlüsselungsstärke}{1}{}

\sse{MKA distributes random SAK using CAK}
\img{img/v3.11.jpg}{}{0.75}{}
\ul
	\li SAK wird für Hashing und Verschlüsselung gebraucht
	\li Hilfsschlüssel KEK werden zur Verteilung gebraucht
	\li Jeder Teilnehmer der Connectivity Association trägt einen Teil zum SAK bei
	\li CAK lebt für Jahre im Switch
	\li KEK wird abgelegt und auch ewig verwendet
	\li 
\ulE

\sse{CAK EAP}
\ul
	\li Statt fest hinterlegt wird der CAK vom EAP Server bezogen
	\li CAK wird von MSK (Master Secret Key) abgeleitet
	\li Key wird periodisch gewechselt \ra Key ist dynamisch
\ulE
\img{img/v3.12.jpg}{Pairwise CAKs usage to to distribute a Group CAK}{0.75}{}


\ch{Applicationsecurity}
\se{OWASP}
\sse{Zugriffsprinzipien}
5A:
\ul
	\li Accounting
	\li Allerting
	\li Auditing
	\li Authenticate
	\li Access Controll
\ulE
\img{img/v4.1.jpg}{}{0.75}{}

\sse{Identifizierung und Authentisierung}
\img{img/v4.2.jpg}{Web Architektur}{0.75}{}
\ul
	\li Http hat kein Gedächnis. Kann die aufeinanderfolgenden Requests nicht in Verbindung bringen. Mit übermittelten Sessionkeys können die Endstationen dies gewährleisten.
	\li Vorne (Tier1-Tier2) weiss man noch wer es ist, beim Zugriff auf Tier 3 wird es dann vergessen! \ra Angriffsmöglichkeit
\ulE

\sse{Sicherheitstechnologien}
% Folie mit Technologien (12) aus Lösungssatz
\ul
	\li TLS: Attacken auf alte Versionen, Falsches Zertifikat erzeugt Warnung, die einfach weggeklickt werden kann
	\li Authentication: What you have: bsp SMS, What you know: bsp Passwort, what you are: bsp Iris
\ulE
\img{img/v4.3.jpg}{OWASP Appsec Tutorial Series Episode 1}{0.75}{}
\exam{WHITEHAT SECURITY WEBSITE STATISTICS REPORT (How Does Your Website Security
	Stack Up Against Your Peers?), Summer 2012, 12th Edition \ra Prüfungsrelevant
	\img{img/v4.4.jpg}{Web Application Security Status}{0.5}{}
}
\exam{Fortsetzung:
	\img{img/v4.5.jpg}{likelihood that at least one vulnerability will appear in a website 	}{0.75}{}
	\img{img/v4.6.jpg}{Window of Exposure (2010)}{0.75}{}
}

\sse{CIA-Sicherheitsprobleme}
\ul
	\li Confidentiality: Jemand kann Daten anschauen auf dem Server, die nicht für ihn bestimmt sind (Passwörter, Domain verwalten, ...)
		\ul
			\li \textbf{Besonders schützenswerte Daten gemäss Gesetz}: Gesundheitsdaten, Rassenzugehörigkeit, Religionszugehörigkeit, Straffälle \ra müssen Verschlüsselt werden
			\li Kreditkarteninformationen fallen NICHT unter das Personendatenschutzgesetz.
			\li Nur Schützenswerte Daten müssen vom Betreiber nicht verschlüsselt werden.
			\li Datenschutz und Personendatenschutz (Person vor Missbrauch dieser Daten schützen) sind zwei komplett unterschiedliche Dinge!
		\ulE
	\li Integrity: 
		\ul
			\li Jemand kann Inhalte auf dem Server verwändern
			\li Nutzung des Webservers als Plattform zum Angriff anderer Rechner bzw. User
		\ulE
	\li Availability: Attacke legt Server lahm
\ulE
\img{img/v4.7.jpg}{ OWASP Top 10 Risk Rating Methodology}{0.75}{}

\sse{Risikoberechnung}
\img{img/v4.8.jpg}{OWASP Threat Agent = Capabilities + Intentions + Past Activities}{0.75}{}

\sss{Verletzlichkeitsfaktoren}
\ul
	\li  Ease of discovery
 	\li Ease of exploit
 	\li Awareness
 	\li Intrusion detection
\ulE

\sss{Technical Impact}
\ul
	\li  Loss of confidentiality
 	\li Loss of integrity
 	\li Loss of availability
 	\li Loss of accountability
\ulE

\sss{Business Impact Factor}
\ul
	\li Financial damage
 	\li Reputation damage (Image, Medien)
 	\li Non-compliance (Verträglich zu Gesetzen und Regeln, Erfüllen von Richtlinien)
 	\li Privacy violation (Personenfaten)
\ulE


\sse{Security Testing Guide}
\img{img/v4.9.jpg}{}{0.5}{}
\img{img/v4.10.jpg}{}{0.5}{}
\img{img/v4.11.jpg}{}{0.75}{}
\img{img/v4.12.jpg}{}{0.75}{}
\img{img/v4.13.jpg}{}{0.75}{}


\ch{Webanwendungssicherheit}
\img{img/v5.1.jpg}{}{0.75}{}
\uli{
	\li Die meissten Angriffe finden heute auf Anwendungsschicht statt.
	\li Keine sprechenden Fehlermeldungen im Frontent ausgeben.
}

\se{Injection}
\expl{Injection}{Jede Beschreibungssprache ist durch Injections verwundbar.}
\img{img/v5.2.jpg}{Protocols}{0.75}{}
\img{img/v5.3.jpg}{SQL Injection}{0.75}{}
\sse{Mögliche Lösungen}
\uli{
	\li Whitelist / Blacklist Patterns (Filtern)
	\li Prepared Statements
	\li Stored Procedures
	\li Escaping: Verhindern, dass Angreifer die Zeichen verwendet, die ihm den Wechsel von den Daten in die Beschreibungssprache ermöglichen.
	\li Least Privileges
}
\expl{Injection ohne Fehlermeldung}{Auch wenn keine Fehlermeldung ausgegeben wird, die hilfreich ist, kann man durch die Aufrufzeit Informationen erhalten \ra Wenn Passwort falsch braucht die Applikation möglicherweise länger als wenn Username falsch oder wenn alles Falsch geht es sehr schnell}


\sse{Wichtige Hash Parameter}
\definition{Hash Funktion}{[pre-Image]\ra(f)\ra[image]}
\uli{
	\li Collision resistence (Zwei beliebige Inputs dürfen nicht den gleichen Hash bilden)
	\li One Way
	\li Pre-Image resistence
	\li Second Pre-Image resistence (Ein zweites finden, dass den gleichen Hash bildet
}


\se{Authentisierung \& Session Management}
\expl{Unterschied Authentisierung und Autorisierung}{Authentisierung: Identitätstest (Was ich bin, habe und weiss), : Autorisierung: Erlaubte Aktionen ermitteln (Autor von welchen Aktionen)}
\definition{Basic Auth}{Browser schickt jedes Mal Zugangsdaten wieder mit. Es gibt keine Session}
\definition{DigestAuth}{Challenge / Respone}
\expl{Challenge Response Funktion}{
	\oli{
		\li Server generiert zufällige Challenge
		\li Client Hasht Passwort+Challenge
		\li Server überprüft Hast
	}
}
\definition{Form-Based Authentisierung}{Klassisch Eingabefeld mit Username/Password}
\definition{TLS-Based Authentitiserung}{Client Verschlüsselt Server Token mit Zertifikat, Server Überprüft Signatur mit Public Key von Client \ra Certificate Chain muss stimmen}

\sse{History-Back}
\uli{
	\li History-Back im Browser ermöglicht einem späteren Nutzer nochmal mit Benutzerdaten des vorherigen Users einzuloggen.
	\li Lösung: Nach Login Redirect vornehmen (Browser merkt sich bei Redirect POST Daten nicht)
}

\expl{Passwort error Delay}{Verzögerung bei falschem Passwort. Angriffsmöglichkeit: Passwort beibehalten und user durchprobieren.}

\sse{Session Attacks}
\uli{
	\li Session Sniffing
	\li Sessin Guessing
	\li Session Cache Stealing
	\li Session Fixation
	\li Session Hijacking
	\li Session Stealing Prevention
}


\ch{Anonymität}
Warum brauche ich Anonymität:
\uli{
	\li Wahrung der Privatsphäre
	\li Redefreiheit
	\li Anonyme Marktforschung
	\li Anonyme Informationssammlung
	\li Whistle-blowing
}

\se{Pseudonym}
\uli{
	\li E-mail Verkehr wird über zwischenstation umgeleitet
	\li Remailer entfernt sämtliche Spuren der ursprünglichen Identität
	\li Problem: Single Point of Vulnerability \ra Wenn Behörde Server beschlagnahmt, ist die Anonymität im Eimer
}
\img{img/v6.1.jpg}{}{0.75}{}

\se{Anonymisierungsnetzwerk (David Chaum's Cascade of Mixes)}
\uli{
	\li Verteilte Pseudonymserver
	\li Anonymisierungsnetzwerk
	\li Der Benutzer wählt \textbf{zufällig} einen Weg durch das Netzwerk \ra nur der Benutzer weis den Weg
	\li über Entry Knoten tritt Benutzerin Netzwerk ein, über Exit aus dem Netzwerk aus
	\li Der Exitknoten wird als erstes von einem Gericht in die Mangel genommen, wenn ein straffälliger Vorfall geschehen war.
	\li Knoten sollten möglichst in unterschiedlichen Ländern sein \ra Erschwert Rechtshilfegesuche
	\li Wenn der Datenverkehr verschlüsselt ist, so läuft ein Rechtshilfegesuch natürlich ins Leere.
}
\img{img/v6.2.jpg}{Benutzer wählt Kette A, E, C}{0.75}{}
\img{img/v6.3.jpg}{destom mehr Verkehr, desto Anonymer ist man unterwegs}{0.75}{}


\sse{Untraceability by using Public Key Cryptography}
\uli{
	\li Nachrichten können bereits an ihrer Meldungslänge erkannt werden \ra Kurze Meldungen werden auf 512 Byte gepadded, lange fragmentiert
}
\img{img/v6.4.jpg}{}{0.75}{}
\oli{
	\li Datenverkehr wird auf dem Client verschlüsselt mit PK des Exit Knoten \ra Exit Knoten kriegt Inhalt mit, wenn der Verkehr nicht End-zu-End Verschlüsselt ist.
	\li Nachricht inkl. Destinationsadresse von Exitknoten wird mit PK von C verschlüsselt
	\li Nachricht inkl. Destinationsadresse von C wird mit PK von E verschlüsselt.
	\li Nachricht inkl. Dest. Addr. von E wird mit PK von Entryknoten verschlüsselt.
}

\sse{Entschlüsselung}
\oli{
	\li A entfernt seine Adresse \ra Frei werdener Platz wird mit Zufallspadding gefüllt, anschliessend wird die Nachricht entschlüsselt
	\li E entfernt wieder seine IP, hängt Padding an und entschlüsselt Inhalt.
	\li ...
}

\sse{Mix KnotenFunktionen}
\img{img/v6.5.jpg}{}{0.5}{}
\uli{
	\li Entschlüsseln des Datenverkehrs
	\li Zeitliches mischen der Anfragen/Ausgaben \ra Eingehende Pakete können nicht mit Ausgehenden korreliert werden, weil sie in zufälliger Reihenfolge rauskomen
	\li Doppelte Pakete löschen (Replay Attacken) \\ Ein Anfreifer könnte ein eingehendes Paket kopieren und nochmal einspeisen. Durch überwachung des Ausgangsverkehr kann er anhand des doppelten Ausgangspaketes das Eingangspaket dem Ausgangspaket zuordnen
}

\sse{Antwort}
\uli{
	\li Retourverschlüsselung sehr komplexs
}

\sse{High Latency Anonymizer}
\uli{
	\li Grosser Puffer
	\li lange Verzögerung
	\li ziemlich sicher
}

\sse{Low-Latency Anonymizers}
\uli{
	\li Anfällig gegen Knotenüberwachung
	\li Man darf nicht zu lange die gleiche Kette bauchen (Tor wechselt häufig die Kette)
	\li Kleine oder gar keine Resorting Puffer
}


\sse{Tor}
\uli{
	\li Ursprünglich von der US Navy entwickelt
	\li Tor kann alle TCP basierten Dienste anonymisieren
	\li Setzt DH Sessionschlüssel ein \ra immer frisch generierte Schlüssel
	\li Es kann beliebig im Pfad ausgestiegen werden \ra Exitknoten weis nicht, das er einer ist
	\li Onion Proxy: Client, der Nur Verkehr ein speist
	\li Onion Router: Knoten
	\li Besser, wenn Cient selber auch Knoten, damit sein eigener Verkehr im Knotenverkehr versteckt ist.
	\li Zwischen den Knoten verwendet Tor TLS
}
\img{img/v6.6.jpg}{}{0.75}{}
\img{img/v6.7.jpg}{Tor Datenformat}{0.75}{}
\uli{
	\li Innerhalb des Netzwerkes gibt es ein standartisiertes Datenformat
	\li Datenverkehr wird auf dem gleichen Pfad wieder zurück gesendet
	\li Stream ID gilt End-zu-End
	\li CircID ermöglicht Zuordnung bei jedem Knoten zwischen Hin- und Retourverkehr
	\li Stream ID weis nur Exit Knoten
}

\img{img/v6.8.jpg}{Circuit}{0.75}{}
\uli{
	\li Stream ist doppelt verschlüsselt (für jeden Knoten einmal)
}

\sse{Anonymer Server}
\img{img/v6.9.jpg}{}{0.75}{}
\uli{
	\li Addresssequence muss in Forum bekannt gegeben werden
	\li User wählt Introduction Point und gibt geschlossenes Paket ab (Addresssequence)
	\li Introduction Point ist wie Exit Point, weiss jedoch um was es geht 
	\li Benutzer wählt Rendezvous Point, dieser hat überhaupt keine Ahnung, um was es geht
	\li Wenn Server antworten möchte, so baut er Verbindung zum Rendezvous Point auf
	\li Datenverkehr läuft über Rendezvous Point
}



\ch{Data Leak Protection}
\sse{Warum Daten schützen?}
\uli{
	\li Vertrauen
	\li Trade Secret (Geheimhaltung)
	\li Rechtliche Aspekte
	\li Bankgeheimnis
	\li Staatssicherheit (Geheimdienst hat je nach Land das Recht in allen Daten zu stöbern)
}

\img{img/v7.1.jpg}{Wo befinden sich Daten in einer Firma}{1}{}
\sse{Data leakage}
	\img{img/v7.2.jpg}{Wo können Daten geleakt werden}{1}{}

	\sse{Data Leak Protection}
	\img{img/v7.3.jpg}{}{0.75}{}
	\uli{
		\li Egress Controll: Was verlässt das Unternehmen
		\li Usage Control: z.B. DRM
	}

\se{Egress Control}
	\img{img/v7.4.jpg}{Datenkontrolle}{1}{}
	\img{img/v7.5.jpg}{Test Daten Anonymisierungsarten}{0.75}{}
	\img{img/v7.6.jpg}{Egress Control Example}{0.75}{}

	\sse{Information Rights Management IRM}
		$IRM = Usage Control + Encryption$ \\
		\ra Controllieren, was mit Informatioin geschieht

		\sse{AD RMS}
		\uli{
			\li Active Directory basiertes Dateisystem-IRM mit MS Office Support
			\li Rechte werden direkt in der Datei gespeichert
		}
		\img{img/v7.7.jpg}{Publish and Consume Files}{0.75}{}


\ch{VPN}

	\definition{VPN}{Virtual Private Network}
	\se{Point to Point Protocol PPP}
		\stdImg{v8.1}{PPP}
		\uli{
			\li  Authentication using PAP (password), CHAP (challenge/response), or the
		Extensible Authentication Protocol (EAP) supporting e.g. token cards
		}
		\stdImg{v8.2}{PPP Encryption Control Protocol}

		\se{Layer2/3/4 Tunneling}
		\stdImg{v8.3}{L2TP Compulsory Mode} 
		\stdImg{v8.4}{L2TP Voluntary Mode}
		\stdImg{v8.5}{Layer3 tunneling IPsec based}
		\stdImg{v8.6}{L2TP  over IPsec - Voluntary Modes}
		\stdImg{v8.7}{Layer4 Tunnel based on TLS}
		\stdImg{v8.8}{L2/3/4 Tunneling Comparison}

	\se{MPLS}
		\definition{MPLS}{Multi Protocol Label Switching}
		\stdImg{v8.9}{MPLS Based VPN}
		\uli{
			\li VPN Paet erhält im ISP Backbone beim Eintritt vor das IP Paket ein Label. Dadurch kann es ohne Auslesen der IP Informationen geroutet werden.
			\li Ein weiteres Label definiert den Routingweg und wird von Hop zu Hop verändert
			\li Label based Routing ist effizienter und ermöglich es einfache statistiken zu erheben
		}
		\stdImg{v8.10}{MPLS L2 Shim Header}
		\uli{
			\li Shim Header is inserted after the DLL Header and before the IP Header (between L2 and L3)
			\li Type Of Service Field ist ähnlich dem Type of Service Field des IP Headers
			\li Das TTL Field übernimmt die gleiche Funktion wie das Time To Live Field des IP Headers
		}

	\se{IPsec Transport Mode}
		\stdImg{v8.11}{IPsec Transport}
		\uli{
			\li IP Datagramms sollten verschlüsselt und authentifiziert sein, damit IP Spoofing und hijacking sowie Modifikationen des Paketes unmöglich sind.
			\li Verschlüsselung ohne Authentication ist verwundbar durch verschiedene Attacken
		}
		\stdImg{v8.12}{IP Authentication Header AH}
		\uli{
			\li Schützt Header und Payload gegen Modifications mittels MAC
			\li Header hat die Struktur einer IPv6 Header Extension, funktioniert allerdinga auch unter IPv4
			\li AH Kann beliebige Transport Layer Protokolle schützen
		}
		\stdImg{v8.13}{IP Encapsulated Secure Payload ESP}
		\uli{
			\li  ESP encrypts the transport payload of the IP datagram using a strong symmetric
		encryption algorithm, (AES, CAMELLIA, 3DES, etc.).
			\li Trailer füllt Paket auf 4 Byte Blöcke auf.
			\li Initialisierungsvektor für die Verschlüsselung, falls ein Paket verloren geht
			\li Struktur einer IPv6 Extension
		}

	\se{IPsec Tunnel Mode}
		\stdImg{v8.14}{VPN}
		\uli{
			\li Connect Subnet's together over unsecure channels
			\li Benutzt Verschlüsselten und Authentifizierten IPsec Tunnel
			\li Hinter dem Secure Gateway können lokale, nicht im Internet addressierbare Adressen verwendet werden, die im Internet nicht addressierbar sind
		}
		\stdImg{v8.15}{VPN Package}
		\stdImg{v8.16}{ESP Header, Payload and Trailer}
		\stdImg{v8.17}{Authenticated Encryption with Associated Data (AEAD)}
		\stdImg{v8.18}{IPsec Tunnel Mode AEAD Packet Overhead}
		\stdImg{v8.19}{IPsec Tunnel using AH}

	\se{Repetition}
		\uli{
			\li Netzwerke über öffentliches Netz koppeln
				\img{img/v8.14.jpg}{VPN}{0.5}{}
			\li private Netzwerkadressen verwenden
			\li Gesammter Verkehr wird getunnelt \ra\ Security Gateway verschlüsselt Pakete und sendet sie aufs Netz, Security Gateway entschlüsselt am Ende die Pakete wieder
			\li Pakete werden durch Tunneln länger als offizielle MTU
			\li Clients merken nichts, dass sie getunnelt werden
			\li IPsec Tunnels sind standardisiert im Gegensatz zu SSL Tunnels
			\li Übertragenes Paket ist verschlüsselt, interner Aufbau nicht sichtbar
			\li Paket enthält am Ende zusätzliche Checksumme, um Integrität zu garantieren
			\li äusserer IP Header ist nicht geschützt
			\li ESP ist eigenständiges IP Protokoll, hat keine Ports
			\li Wann immer möglich AES-GSM als Verschlüsselung verwenden, weil Overhead gering und verschlüsselung schnell
			\li ESP verschlüsselt und sichert normale IP Pakete
		}

	\se{Internet Key Exchange IKE}
		\definition{IKE SA}{Verschlüsselt die IKE UDP Meldungen}
		\definition{IPsec SA}{Verschlüsselung des Payload}
		\uli{
			\li für in/out Verkehr werden unterschiedliche Schlüssel verwendet
			\li Verschlüsselungsalgorithmen werden wie bei TLS ausgehandelt
			\li IKE ist kein Client/Server Protokoll, sondern ein P2P Protokoll (Endpunkte gleichberechtigt)
		}
		\sse{IKEv1}
			\sss{Main Mode}
				\stdImg{v9.1}{IKEv1 Main Mode}
				\uli{
					\li Erste Phase abgeschlossen:
						\uli{
							\li Richtigen Endpunkte verbunden
							\li Verschlüsselte Kommunikation über UDP
						}
				}
				\expl{IKEv1 IPsec Policy}{Legt fest, welche Trafic Selectors gekoppelt werden. Beide Endpunkte müssen die gleiche Policy haben für den QuickMode \ra\ beide Enden müssen einverstanden sein.}
			\sss{Main Mode mit Pre Shared Keys}
				\stdImg{v9.2}{IKE Mainmode using Pre-Shared Keys}
				\uli{
					\li IKE Hashed Benutzerpasswot in $Hash_i$, damit keine Brute Force Attacke auf Hash möglich ist
					\li Problem, VPN Router müsste alle Benutzer durchprobieren, um den richtigen zu finden bei dynamischen IP Adressen. Daher wird der Main Mode nie gebraucht. \ra\ Agressive Mode wird verwendet
				}
			\sss{Agressive Mode mit PSK}
				\stdImg{v9.3}{IKEv1 Agressive Mode}
				\uli{
					\li Problem mit Agressive Mode: Responder übermittelt offen Hash (Hash über beiden Meldungen+Benutzer Passwort)
					\li Passwort Hash kann gesnifft werden und später durch GPU geknackt werden
					\li Wenn PSK schwach sind, kann jeder mit einem Sniffer die Passwörter knacken
				}
				\stdImg{v9.4}{Man-in-the-Middle Attack possible with IKE Aggressive Mode and XAUTH}
				\uli{
					\li VPN Gateway wird nur mit Gruppenpasswort authentifiziert \ra\ Man-in-the-Middle Attack möglich, weil innerhalb einer Firma jeder das Gruppenpasswort kennt.
				}
				\important{Agressive Mode NIE verwenden. Gruppenpasswörter sind für die Katz.}

		\sse{ISAKMP and IPsec Security Associations}
			\stdImg{v9.5}{Rekeying}
			\uli{
				\li Kurz vor Ablauf des Schlüssels, wird ein neuer Key ausgetauscht \ra\ Unterbrechungsreier Schlüsselwechsel
				\li Bei Ablauf der Lifetime wird eine Reauthentisierung gemacht
			}
			
		\sse{IKEv2}
			\uli{
				\li Anzahl Pakete senken
				\li Ist nicht rückwärtskompatibel zu IKEv1
			}
			\stdImg{v9.6}{IKEv2 Authentication and first Child SA}
			\uli{
				\li Nach dem ersten Austausch ist man bereits in der Lage, verschlüsselt fortzufahren
				\li Nur Initiator macht Retransmissions bei Paketverlust, responder macht Acks \ra\ führt zu Robustheit im Vergleich zu IKEv1, bei der beide Retransmit machen und anschliessned zwei Sessions das sind.
				\li Client muss nicht genau Netzwerke / Trafic Selektoren kennen, Gateway wählt automatisch besseren
			}
			
			\sss{Cookie Machanism against DoS Attacks}
				\uli{
					\li Responder sagt dem Initiator, er solle alles mit dem Cookie nochmalsschicken, und berechnet erst dann die DH Key
					\li Im Unterschied zu IKEv1 ist der Cookie Mode nicht mehr standard, sondern muss aktiviert werden
				}
				\stdImg{v9.7}{Cookie Machanism against DoS Attacks}
			
			\sss{Additional CHild SAs}
				\stdImg{v9.8}{Child SA}
				\uli{
					\li Wenn z.B. auf beiden Seiten ein Netz angeschlossen ist, die über das gleiche VPN untereinander verbunden, aber nicht mit den andern werden sollen.
				}

	\se{VPN Applications}
		\stdImg{v9.9}{Abteilungen verbinden, Remote Access}
		
		\sse{Remote Access}
			\stdImg{v9.10}{Remote Access Cases}
			\uli{
				\li dynamische IPs
				\li Virtuelle IPs im Remote Access werden gebraucht, damit Firmenintern der Gateway weiss, das das Paket getunnelt werden muss
			}

	\se{VPN Features}
		\stdImg{v9.11}{Extended Authentication mit EAP}

	\se{NAT Traversal}
		\uli{
			\li Initiator sendet im Hash seine IP mit
			\li Unterscheidet sich beim Responder die IP vom Hash, ist ein NAT zwischendrin
			\li dann wird das IKE Paket in UDP getunnelt, um durch den NAT Router zu kommen
		}
		\stdImg{v9.12}{UDP Tunneling}
		\stdImg{v9.13}{ESP-in-UDP Encapsulation (RFC 3948)}

	\se{Dead Peer Detection}
		\uli{
			\li Jede Minute wird ein Ping gesendet, kommt kein Pong zurück, wird die Verbindung abgebaut
			\li Damit werden Probleme vermieden, wenn Rechner in den Sleep Mode gehen, und sich nicht vom VPN abmelden.
		}



\ch{DNSSEC}
	\se{DNS}
	\expl{Root Server Ddos Attacke}{Nicht mehr realistisch, weil Server redundant und weit verteilt}
	\stdImg{v10.1}{DNS Name Resolution}
		\uli{
			\li Anfrage geht zu ISP
			\li ISP macht rekursive Anfrage
			\li ISP erhält Anwort von Server: ´´Ich bin nicht zuständig, frag XY''
			\li ISP fragt XY
			\li XY liefert wieder andern Server
			\li Dieser liefert IP der Domain
			\li ISP liefert Antwort zurück an Client
			\li Anfrage ist nicht Authentisiert, da Sie nicht vom Client selbst gemacht wurde
		}
		\stdImg{v10.2}{DNS Request}
		\uli{
			\li Rekursiv Flag legt fest, dass der ISP die Abfrage rekursiv machen soll
			\li QID: Query ID
		}
		\stdImg{v10.3}{DNS Response}
		\uli{
			\li Response enthält als additional die IP des zuständigen Nameservers
			\li Diese IP ist nicht authoritativ \ra\ Schwäche
		}
		
		
	\se{DNS Poisoning}	
		\stdImg{v10.4}{DNS Cache Poisoning}
		\uli{
			\li Nameserver zählt QID rauf \ra\ Angreifer kann mit hochgezählter QUI fake Response einspeisen
			\li ISP übernimmt erste Antwort, die unter Umständen Tage im Cache verbleibt \ra\ alle Kunden sind betroffen
		}
		\stdImg{v10.5}{Durch erraten der Query ID und des UDP Source Port kann eine ganze Zone übernommen werden}
		\stdImg{v10.6}{The Dan Kaminsky DNS Vulnerability – July 2008}
		\uli{
			\li Anfrage an Root Server wird abgefangen, indem IP des zuständigen Nameserver eingespiesen wird \ra\ falscher Nameserver
			\li Ermöglicht totale übernahme einer Domäne, inklusive Mix Einträge für Mail, etc.
		}

		\sse{Lösungen}
			\uli{
				\li QID wird randomisiert
				\li Randomisierung der UDP Out-Ports
				\li Trotzdem Angriff durch Brute Force möglich
			}



	\se{DNSSEC}
		\stdImg{v10.7}{DNSSEC Chain of Trust}
		\uli{
			\li Hirarchische Trust Chain
			\li Root Zone hat CA Zertifikat
			\li ICANN hat Root Server zertifiziert \ra\ kein einzelnes Land hat Macht über Root Server
		}
		\expl{Key Signing Key KSK}{Dient zum signieren des Arbeitsschlüssels (Zone Signing Key) \ra\ Key Signing Key (2048bit) wird nur alle paar Monate aus dem Tresor geholt und damit die ZSK (1024bit) für die nächsten Monate signiert}
		\expl{Zone Signing Key ZSK}{Wird verwendet, um alle Records der Zone zu signieren}
		\uli{
			\li .ch wird von Root Zone signiert
			\li Signiert wird ZSK der .ch Zone\ra\ Hash auf KSK der .ch Zone
			\li Vertrauen zwischen .ch Zone und Root Zone wird durch SHA1 Hash, der an die Root Zone gegeben wird und von dieser Signiert wird \ra\ garantiert, dass der .ch-KSK vertrauenswürdig ist.
			\li .ch Zone besitzt wiederum ZSK, der vom KSK signiert ist
			\li KSK der Switch Zone ist wiederum durch .ch-ZSK signiert
			\li switch besitzt wiederum ZSK
			\li Signiert werden nicht einzelne Resource Records, sondern Record Sets (z.B. werden bei Switch alle Nameserver gehashed und dann Signiert)
			\li Bsp Google: Alle Google Domains werden gehashed und Signiert
		}
		\expl{Root Server Vertrauenswürdigkeit}{Wer Kontrolle über den Rootkey erhält, hat die Kontrolle über das gesammte DNS System}
		\expl{Kompromitierter Key}{Kompromitierte Keys können nicht zurückgezogen werden. Laufen einfach nach Laufzeit ab dann gibt es einen neuen\\
		\ra\ TTL der Einträge im Cache muss kleiner sein als die Laufzeit der Zertifikats. Im Normalfall 2 Tage}
		\stdImg{v10.8}{DNS Eintrag}

		\sse{NSEC Einträge /NSEC3}
			\uli{
				\li Wenn Domain nicht existiert, muss die Antwort, dass dieser nicht existiert, auch signiert sein
				\li Durch NSEC Einträge liesse sich eine gesammte Domäne emulieren
				\li NSEC Antwort liefert Range, zwischen definierten Adressen, der keine Domains beinhaltet (z.B. hsr.ch bis hsq.ch)
				\li Es sollte nicht herausfindbar sein, welche Domains existieren
				\li NSEC3: Um zu verhindern, dass bei der Range-Angabe Domains angegeben werden, werden die Domains gehashed, und als Antowrt kommt nur zurück: Zwischen HashA und HashB gibt es keine Domain
				\li Mit einer Wörterbuchattacke lässt sich auch trotz der Hashes ein Grossteil der .ch Domains ermitteln
			}

	
	
	\se{Dane}
		\definition{DANE}{DNS based Authentication of Named Entities}
		\stdImg{v10.9}{DANE TLS Record}
		\uli{
			\li Zertifizierung von Webservern über DNSSEC umsetzen
			\li relativ neue Technologie
			\li DANE definiert neuen Resource Record: TLSA
			\li Browser kann überprüfen, wer eine Seite zertifiziert hat
		}
		
		\sse{Verifying Server and CA Certificates}
			\stdImg{v10.10}{DANE – Verifying Server and CA Certificates}
			\uli{
				\li Bei der DNS anfrage für die Domain (hsr) erhält der Browser auch gleich den TLSA Record, mit dem der Browser überprüfen kann, ob die richtige CA den Server Zertifiziert hat
				\li Dazu wird der Hash von z.B. Quoadis in meinem Zonenfile hinterlegt
			}
			\stdImg{v10.11}{Alternative Verifikation}
		
		\sse{ Getting Server Certificate or Public Key}
			\stdImg{v10.12}{DANE Getting Server Certificate or Public Key}
			\uli{
				\li Root Zertifikat wird mit Dane heruntergeladen
				\li VeriSing wird damit überflüssig, weil jede Domäne ihr eigenes Root Zertifikat erstellen kann und dies einfach in ihr Zonen File eintragen kann
			}
		
		\sse{Verifying Self-Signed Server Certificates}
			\stdImg{v10.13}{Self Signed Server Zertifikate}
			\uli{
				\li Nur derjenige, der das Recht hat, den Eintrag im Zonenfile zu machen, hat auch das Recht, ein Zertifikat zu hinterlegen
				\li Damit werden Root CA's überflüssig
			}
		
		
	\se{Sicherung der Root Zone}
		\stdImg{v10.14}{}
		\uli{
			\li VeriSign zertifiziert den Root Server
			\li Der KSK wird nicht von VeriSign zertifiziert
			\li ICANN zertifiziert den KSK
		}
		\stdImg{v10.15}{}
		
		
\ch{Network Access Controll}		
	\se{Firewall}
		\expl{Firewall}{Firewall braucht es nur, weil die Software so schlecht ist und sich selbst zu schlecht verteidigt \ra\ Firewall ist eine Notlösung}
		\uli{
			\li DMZ: Dienste, die direkt mit dem Internet in Kontakt stehen wie Mailserver, VPN Gateway, ...
			\li Erste Firewalls: Paketfilter
		}
		\stdImg{v11.1}{Aufbau einer dreistufigen Firewall}
		\stdImg{v11.2}{Traffic Control}
		\expl{Statefull Inspection}{Keine statischen Regeln (Paketfilter), sondern Statusüberprüfung (z.B. TCP) anhand dessen genau geprüft werden kann, was für ein Packet gerade erlaubt ist \ra\ dafür können fast wieder alle Ports erlaubt werden. }
		\examp{Statefull Inspection UDP}{DNS Reply ist nur nach DNS Request mit der gleichen QID erlaubt.}
		
	
	\se{NAC}
		\sse{User Authentication}
			\uli{
				\li L2: IEEE 802.1X (swichtes / wlan access points)
				\li L3/4: VPN IKEv2, TLS based methods
			}
			
		\sse{Configuration Assessment}
			\uli{
				\li Client config untersuchen
				\li Clent auf unerwünschte Software prüfen
				\li \ra\ durchsetzen der Policy auf dem Client Device
			}	
			\stdImg{v11.3}{NAC}
			\uli{
				\li NAC Server lässt Client nur zu, wenn sein Gesundheitszustand der Policy entspricht
				\li Client, der nur einen niedrigen Gesunheitszustand besitzt, wird nur ins Isolate Network zugelassen
				\li Drei Stufen: allow, isolate, block
			}
		
		\sse{Trusted Network Connect TNC}
			\stdImg{v11.4}{TNC}
			\uli{
				\li IMC: Bestehender Network Supplicant wird um TNC erweitert und führen Messungen durch
				\li Server überprüft Messungen
				\li Zwischenschicht wird nur als Multiplexer gebraucht
				\li Was ist, wenn der Client (oder ein Trojaner) dem Server Messergebnisse vorgauckelt (Lieing Endpoint)? \ra\ kein Schutz
			}
			
			\stdImg{v11.5}{}
			\uli{
				\li Benutzerauthentisierung über geschützten Kanal (oft EAP)
				\li Anschliessend Gesundheitsmessungen
			}
			\stdImg{v11.6}{RFC Standard}
			
			
			
	\se{Metadata Access Point}
		\uli{
			\li Netzwerkdaten zusammenzuführen und zu korrelieren
			\li Aufgrund dieser Informationen können unerlaubte Zugriffe / geblockt werden
		}
		\stdImg{v11.7}{MAP}
		\stdImg{v11.8}{Extended TNC Architecture}
		\stdImg{v11.9}{Korrelation der Daten}	
			
			
			
\ch{Buffer Overflows}
	\expl{Buffer Overflow}{Angreifer Schickt so viele, entsprechend zusammengestellte Pakete, dass der Buffer des Paketparsers beim Empfänger überläuft und der Code in einen Ausführbaren Bereich überfliesst.}

	\stdImg{v12.1}{Buffer Overflow}
	\stdImg{v12.2}{}
	\stdImg{v12.3}{Segmentation Fault caused by Buffer Overflow}
	\oli{
		\li Stringcopy strcpy() prüft Länge nicht und kopiert einfach das gegebene Argment an die Zieladresse
		\li Mit Stringcopy ist es möglich, in den Bereich des \%ebp und \%eip zu schreiben
		\li Somit ist es möglich, die Rücksprungadresse zu verändern
	}
	\expl{denil of Service Vulnerability}{Server wird abgeschossen durch Buffer Overflow. \ra\ Einfacher als ddos Attacke.}

	\stdImg{v12.4}{execve()}
	\uli{
		\li execve() geht über Interrupts.
		\li Schleusst String mit Pfad zu Shell ein
		\li Buffer beinhaltet Shellcode
		\li Mit noOpperation Code wird die Lücke zwischen Buffer und \%eip gefüllt und am Ende die Rücksprungadresse in \%eip geschrieben.
	}
	\stdImg{v12.5}{The assembly code of execve() starting /bin/sh}
	\stdImg{v12.6}{Using jmp and call to determine string address}
	\stdImg{v12.7}{Null-free Shellcode}
	\stdImg{v12.8}{Including the Buffer Address}
	\stdImg{v12.9}{Making the Buffer Overflow Exploit More Robust}

	\se{Schutzmassnahmen}
		\uli{
			\li Address Space Layout Randomization (ASLR) \ra\ über mehrere hundert Kilobyte variiert der Beginn des Stacks. Der Buffer ist nie so gross, dass man mit nooperation Code diese Lücke überbrücken könnte.\\
			\ra\ Ist keine absolute Hürde, aber erhöht den Schwierigkeitsgrad
			\li Canaries ()
				\uli{
					\li Bekannte Werte werden zwischen Buffer und \%eip wird ein fester Wert reingeschrieben. Vor dem Rücksprung wird dieser Wert auf korrektheit überprüft. (Meisstens werden 0-Character benutzt, weil ein Schadcode keine 0-Character beinhalten kann (0 Terminiert Stringcopy)\\
					\ra\ über mehrere Bufferoverflows kann zuerst die Adresse geändert und anschliessend das Canary wiederhergestellt werden.
					\li Randoom Canary wird verknüpft mit Rücksprungadresse \ra\ sehr schwierig, um Canary wiederherzustellen.
				}
			\li ESP (Executable Space Protecten) \ra\ Flag, der Stack und Heap als Nicht ausführbar flagged \ra\ bei 64Bit Standard
		}


\ch{SmartCards}
	\stdImg{v13.1}{Cryptographical Building Blocks}
	\uli{
		\li Alle Kryptoverfahren benötigen Schlüssel
		\li Schlüssel kann gestohlen werden
	}

	\se{SmartCard Typen}
		\stdImg{v13.2}{SmartCart Typen}
		\expl{Einfache Chipkarten}{Simple Speicherkarten, die einen Wert speichern und es nicht erlauben, denn Wert wieder aufzuladen. Sehr günstig.}
		\examp{Einfache Chipkarten}{Telefonkarten, SIM-Karte}

		\expl{Crypto Card}{Unterstützen RSA. Das BS im ROM ist vor Veränderungen geschützt. Karte blockiert nach 3 Fehlversuchen.}

		\expl{JavaCard}{Karte, deren BS geschützt ist, aber das Profil geändert werden kann. Damit kann die Software darauf aktualisiert werden.}

		\expl{USB Token}{Chipkarte mit USB Schnittstelle. Kein Karten Adapter wird gebraucht. Höhere Datenrate. Teuer.\\
			Günstie Alternative: USB Stick ist nur Adapter für ChipKarte in SIM-Format. Chip muss allerdings einen USB-Kontroller besitzen.
		}
		\expl{Kontaktierung}{Über klassisches Kontaktfeld oder Drahtlos. Bei Drahtloskarten wird der Prozessor über die Luft gespiesen. Reichweite nur wenige Zentimeter. 
			\uli{
				\li Proximity Cards (ISO 14443): distance < 10 cm
	 			\li Vicinity Cards (ISO 15693): distance = 10 cm ... 1 m
	 			\li Mit einer Batterie kommen: bis zu 100Km
			}
		}

		\stdImg{v13.3}{Kartenklassifizierung}


	\se{Display Card}
		\stdImg{v13.4}{}
		\uli{
			\li E-Paper Display
			\li Tasten
			\li Flach Batterie
		}


	\se{NFC}
		\uli{
			\li Bezahlung mit 13Mhz Schnittstelle
		}
		\stdImg{v13.5}{Secure NFC}
		\uli{
			\li NFC Kontroller spricht mit SIM Karte
			\li SWP Singe Wire Protovoll
			\li Soll Schnittstelle vor verseuchtem Smartphone schützen
		}

	\se{Elektrische Kontakte}
		\stdImg{v13.6}{}
		\uli{
			\li Ausgänge 4 und 8 (AUX) früher nicht benutzt, heute für USB Datenleitungen genutzt
			\li Spannungen relativ klein \ra\ Durchschlag
			\li Takt (CLK) wird extern zugeführt (4-5Mhz)
			\li Vcc: Speisung
			\li GND: Masse
			\li SWP: wird für Single Wire Protocoll gebraucht, um z.B. NFC Kontroller anzuschliessen
			\li RST: Reset zur Initialisierung. Gibt Kartentype etc. Zurück
		}

	\se{Physikalische Sicherheit}
		\uli{
			\li Chipkarten BS verhindert das Auslesen der Schlüssel
			\li HSR Karte und SIM Karten machen Challenge Response, Schlüssel wird nie ausgegeben
		}

	\se{Attacken}
		\uli{
			\li Schlüssel könnte auf Bus oder im Ram abgegriffen werden
				\uli{
					\li Schutzschicht auf Chip ist gleichzeitig Dielektrikum für Kondenator. Wird diese weggeäzt um an die Leiterbahnen zu kommen, so veränder der Kondensator seine Kapazität und der Prozessor stellt den Angriff fest.
					\li Security by Obscurity: Die genaue Anordnung auf dem Schip ist unbekannt. Elemente werden vermischt, damit gar nicht genau gewusst wird, wo die Elemente liegen.
					\li Wird die Karte mit Kältespray kaltgestellt, kann Wochen bis Monate später noch der Zustand des RMAs unter dem Elektronenmikroskop ausgelesen werden
					\li Ein Scrambler tauscht die Speicherzellenpositionen im RAM und im EEPROM aus \ra\ wer den Scrambling Algoithmus nicht kennt, kann mit dem Speicherinhalt nichts anfangen.
				}
			\li Stromverbrauchfingerprint zeigt Abfolge von Operationen \ra\ daraus kann der DES Schlüssel abgelesen werden
				\uli{
					\li \stdImg{v13.7}{}
					\li Chip hat keinen Platz für Glättungskondensator
				}
		}


	\se{SmartCard Filesystem}
		\stdImg{v13.8}{SmartCard Filesystem}
		\stdImg{v13.9}{SmartCard File Names}
		\stdImg{v13.10}{SmartCard File Internal Structure}
		
		
	\se{SmartCard Messages}
		\exam{Details über Interner Aufbau um Kommunikation nicht Prüfungsrelevant}
		\stdImg{v13.11}{APDUs – Application Protocol Data Units}
		\uli{
			\li 4Byte Befehle, als Antwort kommen 2Byte Responses
			\li Als Response kann zusätzlich zum ResponseCode z.B. ein Zertifikat mitgegeben werden
		}


	\se{SmartCard Interfaces}
		\uli{
			\li PC/SC
			\li PKCS\#11
				\stdImg{v13.12}{PKCS\#11}
		}

		\exam{PKCS\#15 nicht Teil des Prüfungsstoffes}

			
			
\ch{Plattform Trust}
	\stdImg{v14.1}{How to Establish Trust in a Host and its OS?}
	\uli{
		\li Hardwaremodul, das nicht von Malware unterwandert werden kann, garantiert Echtheit von Health-Messungen des Gerätes
		\li Netzwerkanbieter kann damit verifizieren, das BYOD Gesundheitsstatus korrekt ist
		\li Smartcard könnte auch eine Art TPM darstellen, ist jedoch zu langsam
	}
	\stdImg{v14.2}{TNC Architecture with Platform Trust Service}
	\stdImg{v14.3}{The Future: Trustworthy Virtual Hosts?}
	\uli{
		\li Sichere Anwendung läuft in einer VM, die vom TPM gestartet wurde.
		\li VM beinhaltet z.B. minimal BS mit Ebanking Software
	}
		
					
	\se{TPM}
		\uli{
			\li Chip auf Motherboard fest verlötet
			\li beinhaltet verschiedene Module:
				\stdImg{v14.4}{}
			\li Schwäche von Festplattenverschlüsselung ist Key im RAM. Wenn TPM über Gigabit angebunden wäre, könnte Verschlüsselung darüber laufen
			\li TPM selbst besitzt zwei Schlüssel: Endorsment Key (EK), Storage Root Key (SRK). Alle Geheimnisse hängen am SRK. Wird dieser gelöscht, sind sämmtlliche Dateien nicht mehr lesbar.
			\li PCR: Können nur während dem Booten gelöscht und anschliessend mit neuen Daten über die Hardware gefüllt werden
			\li TPM hängt an LPC Bus, weil es ursprünglich zur Absicherung von Bios/UEFI gedacht wurde \ra\ nur langsame Datenverbindung
				\stdImg{v14.5}{}
			\li Datenschutz
				\uli{
					\li TPM ermöglicht über den EK ein Locking von Software auf eine Gerät
					\li Benutzer kann über EK getrackt werden
					\li Darum wird EK nicht herausgegeben
					\li Benutzer kann aber mittels EK beliebig viele Keys für Anwendungen generieren
						\stdImg{v14.6}{SRK}
				}
			\li Wenn das Motherboard hops geht oder ich einen neuen Rechner kaufe: MigrK (Migration Key) ermöglicht den Wechsel auf eine neue Hardware. MigrK muss jedoch vor dem Verlust des Gerätes erzeugt werden
		}
		
		\sse{Hybrid File Encryption with SRK}
			\exam{Kommt an Prüfung}
			
			\stdImg{v14.7}{}
			\sss{Verschlüsselung}
			\uli{
				\li Effektive Verschlüsselung läuft über normale CPU
				\li TPM generiert Storage Kex (2048bit RSA Schlüssel) \ra\ PriK bleibt in TPM, wird nie im RAM sichtbar, PubK wird auf der Festplatte abgelegt
				\li Symetr. AES Key SymK wird mit StorPubK verschlüsselt
				\li Dateien werden mit AES Key verschlüsselt
				\li SRK verschlüsselt Private Key des TPM und dieser wird dann auf der Festplatte abgelet
			}
			
			\sss{Entschlüsselung}
			\uli{
				\li PriK des SRK entschlüsselt PriK (StorKS), damit wird der AES Schlüssel entschlüsselt
				\li PriK ist nie im normalen RAM der CPU
			}
		
			\important{Baut jeand die Festplatte aus, so ist die Entschlüsselung unmöglich, weil der Entschlüsselungskey des AES Schlüssels durch den PriK des TPM des Originalgerätes geschützt ist.}
		
		\sse{Fields}
			\stdImg{v14.8}{}
		
		\sse{Binding vs. Sealing}
			\stdImg{v14.9}{}
		
			Entschlüsselung eines Dateisystems kann von Werten in TPM Register abhängig gemacht werden. \ra\ BS bootet nicht, wenn sich Werte verändert haben.
			
		\sse{Bootstrap Architecture in PC}
			\stdImg{v14.10}{}
			
			Bevor irgendwas im Bios ausgeführt wird, wird das System gemessen, gehashed, die Hashes addiert und mit den Werten in den TPM Registern (PCR) verglichen
			
			\stdImg{v14.11}{Static Root of Trust for Measurement (SRTM)}
		
			\uli{
				\li PCR Were können auch über TNC übermittelt werden an TNC Server. Bios Rootkit könnte Werte vorlügen. Damit dies nicht geht, müssen die Werte durch das TPM mit dem AIK Schlüssel signiert werden.
				\li Problem: Bis und mit dem Boot Loader kann überwacht werden. BS aber nicht, weil zu komplex und zu viele Module.
				\li Um zu verhindern, dass Malware während dem Laden einer Software ihre eigene Softwarestartet, muss während dem Laden der Software gemessen und im PCR abgelegt werden. Damit kann gewährleistet werden, dass die richtige Software geladen wird.
				\li Damit kann z.B. eine Banking-VM gesichert werden.
			}


\end{document}
